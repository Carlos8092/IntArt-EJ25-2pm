\documentclass{article}
\usepackage{amsmath}
\usepackage{graphicx}
\usepackage{booktabs}

\title{Regresi\'on Logistica en Python}
\author{[Carlos Oswaldo Gonzalez Garza]}


\begin{document}

\maketitle

\section{Introducci\'on}
La regresión logística es una técnica de análisis de datos que utiliza las matemáticas para encontrar las relaciones entre dos factores de datos. Luego, utiliza esta relación para predecir el valor 
de uno de esos factores basándose en el otro. Normalmente, la predicción tiene un número finito de resultados, como un sí o un no.

\section{Metodolog\'ia}
Se siguieron los siguientes pasos para realizar la regresi\'on logistica
\begin{enumerate}
    \item Para comenzar hacemos los Import necesarios con los paquetes que utilizaremos.
    \item Leemos el archivo csv y lo asignamos mediante Pandas a la variable dataframe.
    \item A continuación llamamos al método dataframe.describe() que nos dará algo de información estadística básica de nuestro set de datos.
    \item visualizamos en formato de historial los cuatro Features de
    entrada con nombres “duración”, “páginas”,”acciones” y “valor”. Y también podemos interrelacionar las entradas de a pares, para ver como se concentran linealmente
    las salidas de usuarios por colores: Sistema Operativo Windows en azul, Macintosh en verde y Linux en rojo.
    \item Ahora cargamos las variables de las 4 columnas de entrada en X excluyendo la columna “clase” con
    el método drop(). En cambio agregamos la columna “clase” en la variable y. Ejecutamos X.shape para
    comprobar la dimensión de nuestra matriz con datos de entrada de 170 registros por 4 columnas.
    \item Y creamos nuestro modelo y hacemos que se ajuste (fit) a nuestro conjunto de entradas X y salidas ‘y’.
    \item Una vez compilado nuestro modelo, le hacemos clasificar todo nuestro conjunto de entradas X Y confirmamos cuan bueno fue nuestro modelo
    \item dividimos nuestros datos de entrada en forma aleatoria 
    \item Volvemos a compilar nuestro modelo de Regresión Logística pero esta vez sólo con 80 de los datos de entrada y calculamos el nuevo scoring que ahora nos da 74
    \item Y ahora hacemos las predicciones
    \item También podemos ver el reporte de clasificación con nuestro conjunto de Validación
    \item vamos a inventar los datos de entrada de navegación y lo probamos en nuestro modelo
\end{enumerate}

\subsection{C\'odigo en Python}
\begin{verbatim}
    import pandas as pd
    import numpy as np
    from sklearn import linear_model
    from sklearn import model_selection
    from sklearn.metrics import classification_report
    from sklearn.metrics import confusion_matrix
    from sklearn.metrics import accuracy_score
    import matplotlib.pyplot as plt
    import seaborn as sb
    from sklearn.preprocessing import StandardScaler
    
    # Mostrar gráficas
    plt.show()
    
    # Cargar datos
    dataframe = pd.read_csv(r"usuarios_win_mac_lin.csv")
    
    # Análisis descriptivo
    print(dataframe.head())
    print(dataframe.describe())
    print(dataframe.groupby('clase').size())  # Analizar cuántos resultados hay por clase
    
    # Graficar histogramas
    dataframe.drop(['clase'], axis=1).hist()
    plt.show()
    
    # Pairplot (actualizando parámetro 'size' a 'height')
    sb.pairplot(dataframe.dropna(), hue='clase', height=4, vars=["duracion", 
    "paginas", "acciones", "valor"], kind='reg')
    
    # Preparar datos para el modelo
    X = np.array(dataframe.drop(['clase'], axis=1))  
    y = np.array(dataframe['clase'])
    
    # Escalar los datos
    scaler = StandardScaler()
    X_scaled = scaler.fit_transform(X) 
    
    # Entrenar el modelo
    model = linear_model.LogisticRegression(max_iter=1000)
    model.fit(X_scaled, y)
    
    # Realizar predicciones y mostrar los primeros 5
    predictions = model.predict(X_scaled)
    print(predictions[0:5])
    
    # Validación y evaluación
    model.score(X_scaled, y)
    validation_size = 0.20
    seed = 7
    X_train, X_validation, Y_train, Y_validation = model_selection.train_test_split
    (X_scaled, y,test_size=validation_size, random_state=seed)
    
    # Validación cruzada
    kfold = model_selection.KFold(n_splits=10, shuffle=True, random_state=seed)
    cv_results = model_selection.cross_val_score(model, X_train, Y_train, cv=kfold, 
    scoring='accuracy')
    msg = "%s: %f (%f)" % ('Logistic Regression', cv_results.mean(), cv_results.std())
    print(msg)
    
    # Evaluar en datos de validación
    predictions = model.predict(X_validation)
    print(accuracy_score(Y_validation, predictions))
    print(confusion_matrix(Y_validation, predictions))
    print(classification_report(Y_validation, predictions))
    
    # Realizar predicción para nuevos datos
    X_new = pd.DataFrame({'duracion': [10], 'paginas': [3], 'acciones': [5], 'valor': [9]})
    X_new_scaled = scaler.transform(X_new)  # Escalar los nuevos datos
    print(model.predict(X_new_scaled))  # Realizar la predicción
\end{verbatim}

\section{Resultados}
El modelo de regresi\'on lineal gener\'o los siguientes coeficientes:
\begin{verbatim}
    [2 2 2 2 2]
    0.77647058823529413     
    Logistic Regression: 0.743407 (0.115752)
    0.852941176471
\end{verbatim}

\section{Conclusi\'on}
n este artículo, exploramos cómo construir un modelo de Regresión Logística en Python para clasificar el Sistema Operativo de los usuarios basándonos en sus características de navegación en un sitio web. A partir de este ejemplo, se puede adaptar a otras tareas en las que se necesite clasificar resultados en valores discretos. Si tuviéramos que predecir valores continuos, sería necesario utilizar Regresión Lineal.

\end{document}