\documentclass{article}
\usepackage{amsmath}
\usepackage{graphicx}

\title{Regresión Lineal en Python}
\author{Carlos Oswaldo Gonzalez Garza}

\begin{document}

\maketitle

\section{Introducción}
La regresión lineal es un método estadístico utilizado para modelar la relación entre una variable dependiente y una o más variables independientes. La regresión lineal es un algoritmo de aprendizaje supervisado que se utiliza en Machine
Learning y en estadística. En estadísticas, regresión lineal es una aproximación para modelar la relación entre una
variable escalar dependiente “y” y una o mas variables explicativas nombradas con “X”

\section{Metodología}
Para este experimento, utilizamos Python y la librería scikit-learn para implementar un modelo de regresión lineal. Los pasos seguidos fueron:

\begin{enumerate}
    \item Comenzamos importando las librerías que vamos a usar.
    \begin{verbatim} 
    # Imports necesarios
    import numpy as np
    import pandas as pd
    import seaborn as sb
    import matplotlib.pyplot as plt
    plt.show()
    from mpl_toolkits.mplot3d import Axes3D
    from matplotlib import cm
    plt.rcParams['figure.figsize'] = (16, 9)
    plt.style.use('ggplot')
    from sklearn import linear_model
    from sklearn.metrics import mean_squared_error, r2_score
    \end{verbatim}
    \item Leemos el archivo csv y lo cargamos como un dataset de Pandas y vemos su tamaño.
    \begin{verbatim} 
        #cargamos los datos de entrada
        data = pd.read_csv("./articulos_ml.csv")
        #veamos cuantas dimensiones y registros contiene
        data.shape
        #son 161 registros con 8 columnas. Veamos los primeros registros
        data.head()
        # Ahora veamos algunas estadísticas de nuestros datos
        data.describe()
    \end{verbatim}
    \item Intentaremos ver con nuestra relación lineal, si hay una correlación
        entre la cantidad de palabras del texto y la cantidad de Shares obtenidos.
    \begin{verbatim} 
        # Visualizamos rápidamente las caraterísticas de entrada
        data.drop(['Title', 'url', 'Elapsed days'], axis=1).hist()
        plt.show()
    \end{verbatim}
    \item  Vamos a filtrar los datos de cantidad de palabras para quedarnos con los registros con menos de 3500 palabras y también
    con los que tengan Cantidad de compartidos menos a 80.000. Lo gratificaremos pintando en azul los
    puntos con menos de 1808 palabras y en naranja los que tengan más.
    \begin{verbatim} 
        # Vamos a RECORTAR los datos en la zona donde se concentran más los puntos
        # esto es en el eje X: entre 0 y 3.500
        # y en el eje Y: entre 0 y 80.000
        filtered_data = data[(data['Word count'] <= 3500) & (data['# Shares'] <= 80000)]

        colores=['orange','blue']
        tamanios=[30,60]

        f1 = filtered_data['Word count'].values
        f2 = filtered_data['# Shares'].values

        # Vamos a pintar en colores los puntos por debajo y por encima de la media de Cantidad de Palabras
        asignar=[]
        for index, row in filtered_data.iterrows():
            if(row['Word count']>1808):
                asignar.append(colores[0])
            else:
                asignar.append(colores[1])

        plt.scatter(f1, f2, c=asignar, s=tamanios[0])
        plt.show()
    \end{verbatim}
    \item Vamos a crear nuestros datos de entrada por el momento sólo Word Count y como etiquetas los 
    Shares. Creamos el objeto LinearRegression y lo hacemos encajar con el método fit().
    Finalmente imprimimos los coeficientes y puntajes obtenidos.
    \begin{verbatim} 
        # Asignamos nuestra variable de entrada X para entrenamiento y las etiquetas Y.
        dataX =filtered_data[["Word count"]]
        X_train = np.array(dataX)
        y_train = filtered_data['# Shares'].values

        # Creamos el objeto de Regresión Linear
        regr = linear_model.LinearRegression()

        # Entrenamos nuestro modelo
        regr.fit(X_train, y_train)
        # Hacemos las predicciones que en definitiva una línea (en este caso, al ser 2D)
        y_pred = regr.predict(X_train)

        # Veamos los coeficienetes obtenidos, En nuestro caso, serán la Tangente
        print('Coefficients: \n', regr.coef_)
        # Este es el valor donde corta el eje Y (en X=0)
        print('Independent term: \n', regr.intercept_)
        # Error Cuadrado Medio
        print("Mean squared error: %.2f" % mean_squared_error(y_train, y_pred))
        # Puntaje de Varianza. El mejor puntaje es un 1.0
        print('Variance score: %.2f' % r2_score(y_train, y_pred))
    \end{verbatim}
    \item Por ultimop vamos a intentar probar nuestro algoritmo, suponiendo que quisiéramos predecir cuántos “compartir” obtendrá un articulo sobre ML de 2000 palabras
    \begin{verbatim}
        #Vamos a comprobar:
        # Quiero predecir cuántos "Shares" voy a obtener por un artículo con 2.000 palabras,
        # según nuestro modelo, hacemos:
        y_Dosmil = regr.predict([[2000]])
        print(int(y_Dosmil))    
    \end{verbatim}
\end{enumerate}

El código utilizado fue:

\begin{verbatim}
    # Imports necesarios
    import numpy as np
    import pandas as pd
    import seaborn as sb
    import matplotlib.pyplot as plt
    plt.show()
    from mpl_toolkits.mplot3d import Axes3D
    from matplotlib import cm
    plt.rcParams['figure.figsize'] = (16, 9)
    plt.style.use('ggplot')
    from sklearn import linear_model
    from sklearn.metrics import mean_squared_error, r2_score
    #cargamos los datos de entrada
    data = pd.read_csv("./articulos_ml.csv")
    #veamos cuantas dimensiones y registros contiene
    data.shape
    #son 161 registros con 8 columnas. Veamos los primeros registros
    data.head()
    # Ahora veamos algunas estadísticas de nuestros datos
    data.describe()
    # Visualizamos rápidamente las caraterísticas de entrada
    data.drop(['Title', 'url', 'Elapsed days'], axis=1).hist()
    plt.show()
    # Vamos a RECORTAR los datos en la zona donde se concentran más los puntos
    # esto es en el eje X: entre 0 y 3.500
    # y en el eje Y: entre 0 y 80.000
    filtered_data = data[(data['Word count'] <= 3500) & (data['# Shares'] <= 80000)]
    
    colores=['orange','blue']
    tamanios=[30,60]
    
    f1 = filtered_data['Word count'].values
    f2 = filtered_data['# Shares'].values
    
    # Vamos a pintar en colores los puntos por debajo y por encima de la media de Cantidad de Palabras
    asignar=[]
    for index, row in filtered_data.iterrows():
        if(row['Word count']>1808):
            asignar.append(colores[0])
        else:
            asignar.append(colores[1])
    
    plt.scatter(f1, f2, c=asignar, s=tamanios[0])
    plt.show()
    # Asignamos nuestra variable de entrada X para entrenamiento y las etiquetas Y.
    dataX =filtered_data[["Word count"]]
    X_train = np.array(dataX)
    y_train = filtered_data['# Shares'].values
    
    # Creamos el objeto de Regresión Linear
    regr = linear_model.LinearRegression()
    
    # Entrenamos nuestro modelo
    regr.fit(X_train, y_train)
    # Hacemos las predicciones que en definitiva una línea (en este caso, al ser 2D)
    y_pred = regr.predict(X_train)
    
    # Veamos los coeficienetes obtenidos, En nuestro caso, serán la Tangente
    print('Coefficients: \n', regr.coef_)
    # Este es el valor donde corta el eje Y (en X=0)
    print('Independent term: \n', regr.intercept_)
    # Error Cuadrado Medio
    print("Mean squared error: %.2f" % mean_squared_error(y_train, y_pred))
    # Puntaje de Varianza. El mejor puntaje es un 1.0
    print('Variance score: %.2f' % r2_score(y_train, y_pred))
    #Vamos a comprobar:
    # Quiero predecir cuántos "Shares" voy a obtener por un artículo con 2.000 palabras,
    # según nuestro modelo, hacemos:
    y_Dosmil = regr.predict([[2000]])
    print(int(y_Dosmil))
\end{verbatim}

\section{Resultados}
Los coeficientes obtenidos fueron:
\begin{verbatim}
    Coefficients: [5.69765366]
    Independent term: 11200.303223074163
    Mean squared error: 372888728.34
    Variance score: 0.06
\end{verbatim}


\section{Conclusión}

Implementamos un modelo de regresión lineal para analizar la relación entre la cantidad de palabras en un artículo y sus compartidos. Aunque se encontró una relación positiva, el bajo puntaje de varianza indica que el modelo no explica bien los datos. Esto sugiere que otros factores influyen en la cantidad de compartidos.



\end{document}
