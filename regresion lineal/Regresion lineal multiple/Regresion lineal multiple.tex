\documentclass{article}
\usepackage{amsmath}
\usepackage{graphicx}
\usepackage{booktabs}

\title{Regresi\'on Lineal M\'ultiple en Python}
\author{[Carlos Oswaldo Gonzalez Garza]}


\begin{document}

\maketitle

\section{Introducci\'on}
La regresi\'on lineal m\'ultiple es una t\'ecnica estad\'istica utilizada para modelar la relaci\'on entre una variable dependiente y dos o m\'as variables independientes. En este trabajo, implementamos un modelo de regresi\'on lineal m\'ultiple en Python utilizando la biblioteca \texttt{sklearn}.

\section{Metodolog\'ia}
Se siguieron los siguientes pasos para realizar la regresi\'on lineal m\'ultiple:
\begin{enumerate}
    \item Carga y preprocesamiento de datos.
    \begin{verbatim}
    \end{verbatim}
    \item Definici\'on de variables predictivas.
    \begin{verbatim}
    \end{verbatim}
    \item Entrenamiento del modelo de regresi\'on lineal.
    \begin{verbatim}
    \end{verbatim}
    \item Evaluaci\'on del modelo.
    \begin{verbatim}
    \end{verbatim}
    \item Visualizaci\'on en 3D.
    \begin{verbatim}
    \end{verbatim}
    \item Realizaci\'on de predicciones con nuevos datos.
    \begin{verbatim}
    \end{verbatim}
\end{enumerate}

\subsection{C\'odigo en Python}
\begin{verbatim}
# Importamos librerías necesarias
import numpy as np
import pandas as pd
from sklearn.linear_model import LinearRegression
from sklearn.metrics import mean_squared_error, r2_score
import matplotlib.pyplot as plt
from mpl_toolkits.mplot3d import Axes3D

# Definición de variables predictivas
suma = (filtered_data['# of Links'] + 
        filtered_data['# of comments'].fillna(0) + 
        filtered_data['# Images video'])

dataX2 = pd.DataFrame()
dataX2['Word count'] = filtered_data['Word count']
dataX2['suma'] = suma

XY_train = np.array(dataX2)
z_train = filtered_data['# Shares'].values

# Entrenamiento del modelo
regr2 = LinearRegression()
regr2.fit(XY_train, z_train)

# Coeficientes del modelo
print('Coefficients:', regr2.coef_)
print("Mean squared error: %.2f" % mean_squared_error(z_train, regr2.predict(XY_train)))
print("Variance score: %.2f" % r2_score(z_train, regr2.predict(XY_train)))

# Predicción para un nuevo artículo
z_Dosmil = regr2.predict([[2000, 10+4+6]])
print(int(z_Dosmil))
\end{verbatim}

\section{Resultados}
El modelo de regresi\'on lineal gener\'o los siguientes coeficientes:
\begin{verbatim}
Coefficients: [ 6.63216324 -483.40753769]
Mean squared error: 352122816.48
Variance score: 0.11
\end{verbatim}

Se obtuvo una predicci\'on de aproximadamente 20518 shares para un art\'iculo con 2000 palabras, 10 enlaces, 4 comentarios y 6 im\'agenes.

\section{Conclusi\'on}
Se implement\'o con \'exito un modelo de regresi\'on lineal m\'ultiple en Python, permitiendo estimar la cantidad de "Shares" basada en la cantidad de palabras y la suma de enlaces, comentarios e im\'agenes. A pesar de la mejora respecto al modelo de una sola variable, el puntaje de varianza sigue siendo bajo, indicando que el modelo podr\'ia beneficiarse de m\'etodos adicionales, como la reducci\'on de dimensiones o la inclusi\'on de nuevas variables.

\end{document}
